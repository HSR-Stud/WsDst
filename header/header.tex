%Schriftgr"osse, Layout, Papierformat, Art des Dokumentes
\documentclass[10pt,a4paper,fleqn,headsepline,footsepline]{scrartcl}
%Einstellungen der Seitenränder
\usepackage[left=0.8cm,right=0.8cm,top=0.5cm,bottom=0.5cm,includeheadfoot]{geometry}
% Sprache, Zeichensatz, packages
\usepackage[UTF8]{inputenc}
\usepackage[ngerman]{babel,varioref}
\usepackage{amssymb,amsmath,graphicx,xcolor,lastpage,wrapfig,verbatim}
\usepackage{tabularx}
\usepackage{multicol}
\usepackage{rotating}
\usepackage{floatflt}
\usepackage{array}
\usepackage{scrlayer-scrpage}
\usepackage{multirow,multicol}
\usepackage{trfsigns, trsym}
\usepackage{tikz}
\usepackage{afterpage}


\usepackage[european]{circuitikz}
\usetikzlibrary{patterns}
\pgfdeclarelayer{bg}    % declare background layer
\pgfsetlayers{bg,main}
\newcommand*{\rechterWinkelRadius}{.25cm}
\newcommand*{\rechterWinkel}[2]{% #1 = point, #2 = start angle
   \draw[shift={(#2:\rechterWinkelRadius)}] (#1) arc[start angle=#2, delta angle=90, radius = \rechterWinkelRadius];
   \fill[shift={(#2+45:\rechterWinkelRadius/2)}] (#1) circle[radius=1.25\pgflinewidth];
}


\usepackage{adjustbox}
%
\setkomafont{pageheadfoot}{\footnotesize}
%
%
\RedeclareSectionCommands[
  beforeskip=-.2\baselineskip,
  afterskip=.1\baselineskip
]{section,subsection,subsubsection,paragraph}

\definecolor{pgrey}{rgb}{0.2,0.2,0.2}
\definecolor{black}{rgb}{0,0,0}
\definecolor{red}{rgb}{1,0,0}
\definecolor{white}{rgb}{1,1,1}
\definecolor{grey}{rgb}{0.8,0.8,0.8}
\definecolor{green}{rgb}{0,.8,0.05}
\definecolor{brown}{rgb}{0.603,0,0}

\DeclareMathOperator{\sinc}{sinc}
\DeclareMathOperator{\sgn}{sgn}
\DeclareMathOperator{\Real}{Re}
\DeclareMathOperator{\Imag}{Im}
%\DeclareMathOperator{\e}{e}
\DeclareMathOperator{\cov}{cov}
\DeclareMathOperator{\PolyGrad}{PolyGrad}


\newcommand{\HRule}{\noindent\rule{\linewidth}{1pt}}


\newcommand{\matlab}[1]{\footnotesize{(Matlab: \texttt{#1})}\normalsize{}}
\newcommand{\skript}[1]{${\textcolor{pgrey}{\mbox{\small{Skript S.#1}}}}$}
\newcommand{\kuchling}[1]{$_{\textcolor{red}{\mbox{\small{Kuchling #1}}}}$}
\newcommand{\stoecker}[1]{$_{\textcolor{orange}{\mbox{\small{Stöcker #1}}}}$}
\newcommand{\sachs}[1]{$_{\textcolor{blue}{\mbox{\small{Sachs S. #1}}}}$}
\newcommand{\hartl}[1]{$_{\textcolor{green}{\mbox{\small{Hartl S. #1}}}}$}

%
\newcommand{\myparagraph}[1]{\paragraph{#1}\mbox{}\\\nopagebreak}
\newcommand{\formelbuch}[1]{$\quad{\textcolor{pgrey}{\mbox{\small{S#1}}}}$}

\newcommand*{\diff}{\mathop{}\!\mathrm{d}}

\newcommand{\arraystretchOriginal}{1.5}

\newcommand{\FT}
{
\begin{picture}(1,0.5)
\put(0.2,0.1){\circle{0.14}}\put(0.27,0.1){\line(1,0){0.5}}\put(0.77,0.1){\circle*{0.14}}
\end{picture}
}

\renewcommand{\arraystretch}{\arraystretchOriginal}


\newcolumntype{L}[1]{>{\raggedright\let\newline\\\arraybackslash\hspace{0pt}}m{#1}}
\newcolumntype{C}[1]{>{\centering\let\newline\\\arraybackslash\hspace{0pt}}m{#1}}
\newcolumntype{R}[1]{>{\raggedleft\let\newline\\\arraybackslash\hspace{0pt}}m{#1}}


\author{\authorinfo}
\title{\titleinfo}
%
%Kopf- und Fusszeile
%
\lohead*{\titleinfo}
\rohead*{\today}
\lofoot*{\authorinfo}
\cofoot*{}
\rofoot*{Seite \thepage { }von \pageref{LastPage}}
%
\pagestyle{scrheadings}

