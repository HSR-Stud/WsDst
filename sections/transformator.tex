\section{Transformator}
		
	\subsection{Trafomodelle}
%		\subsubsection{Grundstrucktur des Trafo}
%			\begin{tabular}{p{7cm}p{4.5cm}p{5cm}}
%	        	\textbf{} &
%	        		\begin{minipage}{4.5cm}
%		            	\includegraphics[width=3.5cm]{bilder/GrundstruckturTrafo.png}
%		            \end{minipage} \\
%	        \end{tabular}
		\subsubsection{Der ideale Trafo}
			\begin{tabular}{p{7cm}p{4.5cm}p{5cm}}
				Übersetzungsverh\"altnis &
					$\text{ü} = \frac{|\underline{U}_1|}{|\underline{U}_2|} =
					\frac{|\underline{I}_2|}{|\underline{I}_1|} = \frac{N_1}{N_2}$ &
					\begin{minipage}{4.5cm}
						\includegraphics[width=3.5cm]{bilder/GrundstruckturTrafo.png}
					\end{minipage} \\
				Leistungsbilanz &
					$\underline{S}_1 = \underline{S}_2$ &
					oder $\underline{U}_2 \cdot \underline{I}_2^* = \underline{U}_1 \cdot \underline{I}_1^*$ \\
				Scheinwiderstandsübersetztung &
					$\underline{Z}_{aU} = \frac{|\underline{U}_1|^2}{|\underline{U}_2|^2} \cdot \underline{Z}_a = \text{ü}^2 \cdot \underline{Z}_a$ &
					\begin{minipage}{4.5cm}
	            		\includegraphics[width=3.5cm]{bilder/IdealerTravoImpedanzwandler.png}
	            	\end{minipage}
			\end{tabular}
		%TODO: 02.02.12 DK, Newpage wieder entfernen wen möglich 
		\newpage
		
		\subsubsection{Verlustloser und Streuungsfreier Trafo}
			\renewcommand{\arraystretch}{1.5}
\textbf{Induktionsgesetz}\\
		\begin{tabular}[c]{p{8.7cm}}
			$u_i= \mp \dot{\Phi} = \mp \frac{d}{dt} \int \vec{B} \cdot
			\vec{dA}\qquad $ \parbox{3cm}{\tiny{$- \Rightarrow B,u_i $
			Rechtsschraube\\ $+ \Rightarrow B,u_i $
			Linksschraube}}
			\\
			
			$u_i= \mp \dot{\Psi}\qquad$ , meist $\; u_i = \mp
			N\cdot\dot{\Phi}$		
		\end{tabular}
		\parbox{8cm}{Durchsetzt das sich \"andernde Magnetfeld einer stromdurchflossenen Spule
					eine zweite Spule, so wird auch in dieser eine Spannung
					(=Gegeninduktionsspannung) induziert.}	

\begin{multicols}{2}
 		\textbf{Gegeninduktion} ($M_{\textcolor{blue}{X}\textcolor{orange}{Y}}; \textcolor{blue}{X}$: Wirkung,
 		$\textcolor{orange}{Y}$: Ursache)\\
		\begin{tabular}{ll}
  		Gegeninduktivit\"at
  			& $M_{21} = \frac{\Psi_{m21}}{i_1}$ Meist $= \frac{N_2 \phi_{m21}}{i_1}$\\ 
  			(wenn $\mu$ = const.) & $M = k \cdot \sqrt{L_1 L_2} = M_{21} = M_{12} $  \\
  			Gegeninduktionsspannung
  			& $u_{21} = \dot{\Psi}_{21} = M_{21} \frac{di_1}{dt}$ \\
		\end{tabular}\\

  		\textbf{Transformatorgleichungen}\\
		$\boxed{u_1 = L_1 \dfrac{di_1}{dt} \textcolor{red}{+}\textcolor{green}{-}M_{12}
		\dfrac{di_2}{dt} = L_1 \dfrac{di_1}{dt} \textcolor{red}{-}\textcolor{green}{+}M_{12} \dfrac{di_b}{dt}}$ \\
		$\boxed{u_2 = L_2 \dfrac{di_2}{dt}\textcolor{red}{+}\textcolor{green}{-} M_{21}
		\dfrac{d i_1}{dt} = -L_2 \dfrac{di_b}{dt}
		\textcolor{red}{+}\textcolor{green}{-} M_{21} \dfrac{d i_1}{dt}}$\\
		Im Bildbereich:\\
		$\underline{U}_1 = j\omega\cdot L_1 \cdot \underline{I}_1 + j\omega\cdot M \cdot \underline{I}_2$\\
		$\underline{U}_2 = j\omega\cdot L_2 \cdot \underline{I}_2 + j\omega\cdot M \cdot \underline{I}_1$\\
  			
	  	\textbf{Idealer Trafo}\\ 
	  	$"u = \dfrac{u_1}{u_2} = \dfrac{N_1}{N_2} = \sqrt{\dfrac{L_1}{L_2}}$ $\qquad$ (im Leerlauf:
	  	$\dfrac{1}{"u} = k \sqrt{\dfrac{L_2}{L_1}}$)\\

  		\textbf{Verlustbehafteter Trafo}
  		\begin{list}{$\bullet$}{\setlength{\itemsep}{0cm} \setlength{\parsep}{0cm} \setlength{\topsep}{0cm}} 
          \item Prim\"arstrom im Leerlauf: $L_H$
          	(ideal $L_H \rightarrow \infty)$
          \item Hysterese- \& Wirbelstromverluste: $R_{Fe}$ \newline
          	(ideal: $R_{Fe}\rightarrow \infty$) 
          \item Kupferwiderst\"ande: $R_{Cu1}, R_{Cu2}$
          	(ideal: $R_{Cu}
          \rightarrow 0$)
          \item Streufluss (Kopplung): $L_{\sigma1}, L_{\sigma2}$
          	(ideal: $L_{\sigma} \rightarrow 0$)
        \end{list}

	\columnbreak
  		\begin{flushleft}
  		\includegraphics[width=4cm]{bilder/trafo-kopplung.png} \\
  		\small{\textcolor{red}{Gleichsinnig} / \textcolor{green}{Gegensinnig}} \\
  		\vspace{1cm}

	  	\includegraphics[width=5cm]{bilder/T_Ersatzschaltbild_VST.png}\\
		\end{flushleft}  	
	
\end{multicols}
\renewcommand{\arraystretch}{1} 	
		\subsubsection{Transformatoren-Hauptgleichung (gilt bei Leerlauf)}
			\begin{tabular}{p{7cm}p{4.5cm}p{5cm}}
      			$| \hat{u}_{10} | = \hat{i}_0 \cdot X_{L1}$
      			& 	$\hat{u}_{10} = \hat{i}_0 \cdot \omega \cdot L_1$ \\
      		
				$U_{20} = \frac{2\pi}{\sqrt{2}}N_2 f \hat{B}_1 A$
				&
            	$U_{10} = \frac{2\pi}{\sqrt{2}}N_1 f \hat{B}_1 A$ &
					wobei $\frac{2\pi}{\sqrt{2}} = 4.44$ und $\hat{B} \cdot A = \hat{\Phi}$ 
			\end{tabular}
	\subsection{Der reale (einphasige) Transformator}
	Für Dreiphasige Transfromatoren bei die Spannung mit dem Faktor $\sqrt(3)$
	teilen, bei Leistungen  Strom und Spannung mit dem Faktor $3$
	Multiplizieren um alle drei Phasen abzudecken.\\
		\subsubsection{Ersetzen der magnetischen Kopplung}
			\begin{tabular}{p{5.8cm}p{8cm}p{4.5cm}}
            	Spannung, Strom &
            		$U_2' = U_2 \cdot \frac{N_1}{N_2}$ \quad 
            		$I_2' = I_2 \cdot \frac{N_2}{N_1}$ \\
            	Nennstrom & 
            		$I_N = S_N / U_N$
            	Widerstand, Streufluss &
            		$R_2' = R_2 \cdot (\frac{N_1}{N_2})^2$ \quad 
            		$X_{\sigma 2}' = X_{\sigma 2} \cdot (\frac{N_1}{N_2})^2$ \\
            	Vollst\"andiges Ersatzschaltbild &
            		\begin{minipage}{8cm}
	            	
	            		\includegraphics[width=6cm]{bilder/VollstaendigesErsatzschaltbild.png} \end{minipage} &
					\begin{minipage}{4.5cm}
                    	\tiny
                    		$R_1, R_2'$: Widerstand Spule\\ \\
                    		$jX_{\sigma 1}, jX_{\sigma 2}'$: Streufluss Spule\\ \\
                    		$R_{Fe}$: Eisenverlust\\ \\
                    		$jX_h$: Hauptfluss Spule\\
                    \end{minipage} \\ \\
				Systemgleichung des realen Trafo &
					$\underline{U}_1 = R_1\cdot\underline{I}_1 + jX_{\sigma 1}\cdot\underline{I}_1 + jX_h\cdot(\underline{I}_1+\underline{I}_2')$ \\
					& $\underline{U}_2' = R_2'\cdot\underline{I}_2' + jX_{\sigma 2}'\cdot\underline{I}_2' + jX_h\cdot(\underline{I}_1+\underline{I}_2')$

            \end{tabular}
            	Für mittlere Transformatorenleistungen erbgeben sich etwa folgende Relationen zwischen
            	den einzelnen $R$s und $X$s:
            	$$ R_1 : R_2' : X_{\sigma 1} : X_{\sigma 2}' : X_h : R_{Fe} \approx 1:1:2:2:1000:10000	$$
            	
		\subsubsection{Leerlauf und Magnetisierung}
			\begin{tabular}{p{5cm}p{6cm}p{7cm}}
            	Rechnen &
            		\begin{minipage}{13cm}
                    	Mit Leerlaufdaten $R_{Fe}$ und $X_h$ ausrechnen. $R_1$ und $X_{\sigma1}$ vernachl\"assigen.
                    \end{minipage} \\ \\
            	Induktiver Leerlaufstrom &
            		$\underline{I}_{10} = \underline{I}_{1Fe} + \underline{I}_{1\mu}$ $(\underline{I}_{1\mu} \gg \underline{I}_{1Fe})$ &
            		\begin{minipage}{8cm}
	            		\includegraphics[width=5cm]{bilder/ErsatzschaltbildTrafoLeerlauf.png}
	            	\end{minipage} \\ \\
				Magnetisierungsstrom &
					$i_\mu = \sqrt{2}I_{\mu1}\cdot \sin(\omega t) + \sqrt{2}I_{\mu3}\cdot \sin(3\omega t) + \sqrt{2}I_{\mu m}\cdot \sin(m\omega t)$ &
					\hspace{3.3cm}$(m=2n+1 \hspace{0.3cm} n\epsilon \mathbb{N}_0)$ \\
					&
					Für den Effektivwert des Stromes: &
					$I_{\mu RMS} = \sqrt{I_{\mu1}^2 + I_{\mu3}^2 +\ldots+ I_{\mu m}} $\\ \\
				Leerlaufverluste &
					$P_0 = P_{0Cu} + P_{0Hy} + P_{0Wi}$ \\
				Hystreseverluste &
					$P_{Hy} \sim f \cdot B^2$ \\
				Wirbelstromverluste &
					$v_W = c_W \cdot f^2 \cdot B^2$ &
					$c_W$ ist materialabh\"angige Konstante \\
				Relativer Leerlaufstrom &
					$i_{0N} = \frac{I_{0N}}{I_{1N}}$ &
					$I_{1N}$ ist eingangsseitiger Nennstrom \\
				Eisenverluststrom &
					$I_{Fe} = \frac{P_{0N}}{U_{1N}} = I_0 \cdot \cos(\varphi_0)$ \\
				Eisenverlustwiderstand &
					$R_{Fe} = \frac{U_{1N}^2}{P_{0N}} = \frac{P_{0N}}{I_{Fe}^2}$ \\
				Hauptreaktanz &
					$X_h = L_h \omega = \frac{U_{1N}}{I_{\mu}} = \frac{U_{1N}^2}{Q_{0N}}
					=
					\frac{Q_{0N}}{I_{\mu}^2}$
					& $Q_{0N} = \sqrt{S_{0N}^2 - P_{0N}^2}$ \\
				Magnetisierungsstrom &
					$I_\mu = I_{0N} \cdot \sin(\varphi_0) = \sqrt{I_0^2 - I_{Fe}^2}$&
					\begin{minipage}{7cm}
                    	\begin{tabular}{p{2.5cm}p{3.5cm}}
	                    	\begin{minipage}{2.5cm}
	                        	\includegraphics[height=1.5cm]{bilder/ZeigerdiagrammRealerTrafoLeerlauf.png}
	                        \end{minipage} &
							\begin{minipage}{3.5cm}
	                       		oder $\hat{I}_{\mu} = \frac{\hat{H}_{Fe} \cdot l_{Fe}}{N_1}$
	                        \end{minipage}
						\end{tabular}
	            	\end{minipage} \\
				Leistungsfaktor im Leerlauf &
					$\cos(\varphi_0) = \frac{P_{0N}}{I_{0N} \cdot U_{1N}}$ \\
            \end{tabular}
		\subsubsection{Kurzschluss}
			\begin{tabular}{p{6cm}p{6cm}p{6.5cm}}
            		\begin{minipage}{12cm}
                    	Mit Kurzschlussdaten $R_1$ und $X_{\sigma1}$ ausrechnen. $R_{Fe}$ und $X_h$ vernachl\"assigen.
                    \end{minipage} 
					& &	
					\multirow{3}{6.5cm}{
					%\begin{minipage}{7cm}
	            		\includegraphics[height=2.5cm]{bilder/KurzschlussErsatzschaltbild.png} } \\
	            	%\end{minipage}} \\
	            \\
				Kurzschlussimpedanz&
					$\underline{Z}_k = R_k + jX_k$ \\
					& $Z_k = \frac{U_k}{I_k}$ \\
					& $R_k = R_1 + R_2' = \cos{\varphi_k} \cdot Z_k  = \frac{P_k}{I_k^2}$  \\
					& $X_k = X_{\sigma1} + X_{\sigma2}' = \sin{\varphi_k} \cdot Z_k = \frac{Q_k}{I_k^2}$ \\
				Leistungsfaktor im Kurzschluss&
					$\cos(\varphi_k) = \frac{P_k}{U_k \cdot I_k} = \frac{P_k}{S_k}$ \\
             \end{tabular}

		\subsubsection{Spannungs\"anderung bei Belastung}
			\begin{tabular}{p{10cm}p{6cm}}
            		\begin{minipage}{10cm}
	            		$\underline{U}_1 =
	            		\underline{U}_R+\underline{U}_X+\underline{U}_2'$ \qquad
	            		$\underline{U}_2'=\underline{U}_2 \cdot "u$\\ \\
	            		$\underline{U}_R=R_k \cdot \underline{I}_1$ \qquad
	            		$\underline{U}_X=jX_k \cdot \underline{I}_1$\qquad
	            		$\underline{I}_2' = \underline{I}_1$\\
	            		\end{minipage} &
            		\begin{minipage}{6cm}
	            		\includegraphics[width=5cm]{bilder/ErsatzschaltbildTrafoLast.png}
	            	\end{minipage}\\			
            \end{tabular}\\ \\

			\begin{tabular}{p{8cm}|p{10cm}}
	 			\textbf{Zeigerdiagramme} & \textbf{Kappsches Dreieck}\\
				\begin{minipage}{8cm}
	            	\includegraphics[width=8cm]{bilder/KappschesDreieck_1.png}
	            \end{minipage}  & \hspace{0.2cm}
				\begin{minipage}{10cm} 
		        	\begin{minipage}{2.5cm}
						\includegraphics[width=3cm]{bilder/KappschesDreieck.png}
		            \end{minipage}
					\begin{minipage}{7.5cm}
		      			$$u_{\varphi} = u_{\varphi'} + 1 - \sqrt{1 - u_{\varphi''}^2}$$
		      			$$ u_\varphi \approx u_{\varphi'}\quad (\text{für }u_k
		      			=\frac{U_k}{U_{N}} \cdot 100\% < 4 \%)$$
		      			$$u_{\varphi'} = u_r \cdot \cos \varphi + u_x
		      			\cdot \sin \varphi$$ $$u_{\varphi''} = u_x \cdot \cos \varphi - u_r \cdot \sin \varphi$$
		      			$$ u_r = \frac{R_k I}{U_N} \qquad u_x = \frac{X_k I}{U_N} $$
		      			$$ U_2 = \frac{U_N}{"u} - \Delta U_2 = \frac{U_1}{"u} - \frac{1}{"u}
		      			U_N u_\varphi $$\\
		      		\end{minipage}         
                \end{minipage}\\
				\begin{minipage}{8cm}
 					\begin{tabular}[c]{p{2.66cm}p{2.66cm}p{2.66cm}}
                     	$\quad \varphi = 0$ & $\quad\varphi > 0$ & $\quad\varphi
                     	< 0$\\ rein ohmsche & induktive & kapazitive\\
                     	Last & Last & Last\\
                     	&&\\
                     	&&\\
                     	&&\\
                     	&&\\
                     \end{tabular}               
                \end{minipage}& \hspace{0.2cm}
				\begin{minipage}{10cm}
		      		Beim kappschen Dreieck wird mit ``genormten'' Gr\"ossen (klein $u$) gerechnet: 
		      		$u = \frac{U_{Strang}}{U_{N, Strang}} = \frac{U_X}{U_1} \qquad
		      		\boldsymbol{U_N = U_1} $\\ \\ Das Kappsche Dreieck dreht sich um die
		      		Spitze der Prim\"arspannung. Bei konstantem Strom und variablem $cos(\varphi)$ beschreibt ${u}_2'$
		      		einen
		      		Kreis um die Prim\"arspannung.\\ Bei \textbf{kapazitiver Last steigt}
		      		die Sekund\"arspannung über den Leerlaufwert an. \\                  
                \end{minipage}     
            \end{tabular}

		
		\subsubsection{Wirkungsgrad des Trafos}
			\begin{tabular}{p{5cm}p{7cm}p{7cm}}
            	Wirkungsgrad &
            		$\eta = 1-\dfrac{P_{V0} + P_{VK} \cdot
            		(\frac{P_B}{P_N})^2}{P_B} $ &
            	\begin{minipage}{7cm}
                	$P_B$ = Betriebsnennleistung\\
                	$P_{N}$ = Nennleistung\\
                	$P_{V0}$ = Leerlaufverlustleistung\\
                	$P_{VK}$ = Kurzschlussverlustleistung                	
                \end{minipage}\\ \\
            	 &
            		$\eta = 1 - \dfrac{a + (\frac{P_B}{P_N})^2}{P_B} P_{VK}$ &
            	\begin{minipage}{7cm}
 					$a = \dfrac{P_{V0}}{P_{VK}}$                	
                \end{minipage}\\ \\            		
            	 &
            		$\eta = 1 - \frac{P_{V0}+P_{VK}}{P_B} $ 
            		& (bei Wirkleistungsvollast) \\ \\
            	Maximaler Wirkungsgrad
            	& $S_{\eta-max} = \sqrt{a} \cdot S_N$ $P_{\eta-max} = \sqrt{a}
            	\cdot P_N$
            	& (Kupferverluste = Eisenverluste)
            \end{tabular}	

\newpage
	\subsection{Transformatoren Kühlmittel}
		\begin{center}
	    	\includegraphics[width=19cm]{bilder/Kuehlmittel.png}
	    \end{center} 

	\subsection{Drehstrom-Leistungstransformatoren} 
	Angegebene Leistung bezieht sich immer auf alle drei Wicklungen. $\Rightarrow P_{Wicklung} =
	\frac{P_N}{3}$ \\
	Angegebene Spannungen und Str\"ome gelten immer für Aussenleiter. Somit muss immer entweder - je
 nachdem ob Dreieck- oder Sternschaltung vorliegt -	Strom oder Spannung mit Faktor
 $\frac{1}{\sqrt{3}}$ multipliziert werden.
 	\subsubsection{Bauformen}
	 $$\text{Kennzeichnung} = [a][b][c][d] = \begin{cases}
                  [a] = \text{Oberspannungswicklung, Grossbuchstabe (Y,D,III,Z)
                  }\\
                  [b] = \text{Unterspannungswicklung, Kleinbuchstabe (y,d,iii,z) } \\
                  [c] \cdot 30^\circ = \text{Phasenverschiebung zwischen Unter- und Oberspannung }
                  \\ [d] = 0 \text{, falls Neutralleiter herausgeführt (optional)}
                  \end{cases}$$
		\begin{center}
	    	\includegraphics[height=6cm]{bilder/Drehstromtrafo.png}
	    \end{center} 
