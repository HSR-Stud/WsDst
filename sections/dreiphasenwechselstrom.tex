\section{Dreiphasenwechselstrom (Drehstrom)}
	%\subsection{Entstehung des Drehstrom-Systems}
		\begin{tabular}{p{8.5cm}p{9cm}}
        	\begin{minipage}{8cm}
            	\includegraphics[width=7.5cm]{bilder/Drehstrom.png}
            \end{minipage} &    
			\begin{minipage}{10cm}
            	Zeiger drehen mit $\omega t$ im Gegenuhrzeigersinn ($\omega > 0$). \\
            	$\underline{U}_2$ ist gegenüber $\underline{U}_1$ 
				$120^{\circ}$ nacheilend, $\underline{U}_3$ gegenüber $\underline{U}_1$ $240^{\circ}$.  \\ \\
				Somit gilt (bei symmetrischer Belastung): \\
				$\underline{U}_2 = \underline{U}_1 \cdot e^{j (-120^{\circ})} \qquad \underline{U}_3
				= \underline{U}_1 \cdot e^{j (-240^{\circ})} = \underline{U}_1 \cdot e^{j
				(120^{\circ})}$\\
				$\underline{U}_1 +\underline{U}_2 + \underline{U}_3 = 0 \qquad \underline{I}_1 +\underline{I}_2 + \underline{I}_3 = 0$
            \end{minipage}
        \end{tabular}
		
% 		\subsubsection{Verketten der Spannungen oder Str\"ome}
% 			\begin{tabular}{p{5cm}p{6cm}p{7cm}}
%             	\textbf{Maschensatz} &
%             		\fbox{$\underline{U}_1 + \underline{U}_2 + \underline{U}_3 = 0$} \\
%             	\textbf{Knotenpunktsatz} &
%             		\fbox{$\underline{I}_1 + \underline{I}_2 + \underline{I}_3 = 0$} \\
%             \end{tabular}

		\subsubsection{Stern- (Y) / Dreieckschaltung ($\Delta$)}
            	\renewcommand{\arraystretch}{1.5}
			\begin{tabular}{| p{4.5cm} | l | l |}
				\hline
	 				& Sternschaltung (Y)		& Dreieckschaltung ($\Delta$)\\
	 			\hline
	 			\vspace{0.2cm}
	 				\textbf{Rechtsdrehend:} 
	 				&
	 					\parbox[c][3.5cm][c]{5cm}{\includegraphics[width=5cm]{bilder/Sternspannung.png}} &
	 					\parbox[c][3.5cm][c]{5cm}{\includegraphics[width=5cm]{bilder/Dreieckstrom.png}} \\
						
					%Phasenverschiebung &
					%	$\underline{U}_{12} = 400V; \underline{U}_{23} = 400Ve^{-j120^\circ}; \underline{U}_{31} = 400Ve^{j120^\circ}$&
					%	\\
		 			Verkettete Spannung &
		 				$U = U_{Str} \cdot \sqrt{3} \qquad \underline{U} = \underline{U}_{Str} \cdot \sqrt{3} \cdot e^{j 30^\circ}$ &
		 				%$U = U_{Str}$ \hspace{0.2cm}
		 				$\underline{U} = \underline{U}_{Str}$ \\
		 			Aussenleiterströme &
		 				%$I = I_{Str}$ \hspace{0.2cm}
		 				$\underline{I} = \underline{I}_{Str}$ &
		 				$I = I_{Str} \cdot \sqrt{3}  \qquad \underline{I} =
		 				\underline{I}_{Str} \cdot \sqrt{3} \cdot e^{-j 30^\circ} $ \\
					Gesamt-Scheinleistung &
						\multicolumn{2}{c|}{$S = 3 \cdot S_{Str} =\sqrt{3} \cdot U \cdot I $ \hspace{0.2cm} in [VA]}\\
		 			Scheinleistung pro Strang &
						\multicolumn{2}{c|}{ 
						 %$S_{Str} = U_{Str} \cdot I_{Str} \qquad
						  $\underline{S}_{Str} = \underline{U}_{Str} \cdot \underline{I}_{Str}^\ast$ \hspace{0.2cm} in [VA] Strom konjugiert komplex!!}  \\
		 			Wirkleistung &
		 				\multicolumn{2}{c|}{ $P = S \cdot \cos\varphi = \sqrt{3} \cdot U \cdot I \cdot \cos\varphi$ \hspace{0.2cm} in [W]} \\
		 			Blindleistung &
		 				\multicolumn{2}{c|}{ $Q = S \cdot \sin\varphi = \sqrt{3} \cdot U \cdot I \cdot \sin\varphi$ \hspace{0.2cm} in [var]} \\
 		 			Wirkarbeit &
 		 				\multicolumn{2}{l|}{\hspace{3cm} $W = P \cdot t = \sqrt{3} \cdot U \cdot I \cdot cos\varphi \cdot t$ \hspace{0.2cm} in $[Ws, kWh]$} \\
% 		 				&\multicolumn{2}{c|}{}\\
 		 			Blindarbeit &
 		 				\multicolumn{2}{l|}{\hspace{3cm} $W_b = Q \cdot t = \sqrt{3} \cdot U \cdot I \cdot sin\varphi \cdot t$ \hspace{0.2cm} in $[varh, kvarh]$} \\
	 			\hline
			\end{tabular}
        \renewcommand{\arraystretch}{1}
		
		%\subsubsection{Bestimmung des Y-Punktes mittels Leitwert-Operatoren im Vierleiter-Drehstromnetz}
        \subsubsection{Ausfall des Neutralleiters: Bestimmung des Y-Punktes mittels Leitwert-Operatoren}
            \begin{tabular}{p{5cm}p{13cm}}
            	\begin{minipage}{8cm}
                	\includegraphics[width=5cm]{bilder/ZeigerdarstellungVerschobenerSternpunkt.png}
                \end{minipage} &
				\begin{minipage}{13cm}
					\begin{center}
						$ \underline{U}_{12}; \underline{U}_{23}; \underline{U}_{31} \Rightarrow const$
					\end{center}
                	$\underline{U}_{12} = \underline{U}_{1N'} - \underline{U}_{2N'}$ \hspace{0.3cm}
                	$\underline{U}_{1N'} = \underline{U}_{1N} + \underline{U}_{NN'}$ \hspace{0.3cm}
                	$\underline{I}_1 = \underline{Y}_1 \cdot \underline{U}_{1N'} = \underline{Y}_1 \cdot (\underline{U}_{1N} + \underline{U}_{NN'})$ \\
                	$\underline{U}_{23} = \underline{U}_{2N'} - \underline{U}_{3N'}$ \hspace{0.3cm}
                	$\underline{U}_{2N'} = \underline{U}_{2N} + \underline{U}_{NN'}$ \hspace{0.3cm}
                	$\underline{I}_2 = \underline{Y}_2 \cdot \underline{U}_{2N'} = \underline{Y}_2
                	\cdot (\underline{U}_{2N} + \underline{U}_{NN'})$ \\ $\underline{U}_{31} = \underline{U}_{3N'} - \underline{U}_{1N'}$ \hspace{0.3cm}
                	$\underline{U}_{3N'} = \underline{U}_{3N} + \underline{U}_{NN'}$ \hspace{0.3cm}
                	$\underline{I}_3 = \underline{Y}_3 \cdot \underline{U}_{3N'} = \underline{Y}_3
                	\cdot (\underline{U}_{3N} + \underline{U}_{NN'})$ \\ \\ 
                	$$\underline{U}_{NN'} = \boldsymbol{-} \frac{(\underline{Y}_1 \cdot
                	\underline{U}_{1N} + \underline{Y}_2 \cdot \underline{U}_{2N} + \underline{Y}_3 \cdot
                	\underline{U}_{3N})}{\underline{Y}_1 + \underline{Y}_2 +
                	\underline{Y}_3}$$
                \end{minipage}
			\end{tabular}

%		\subsubsection{Anwendung der Y- und $\Delta$- Schaltung}
	
	\subsection{Stern-Dreieck-Umwandlung}% \formelbuch{18}}
	%\begin{figure}
	  \begin{minipage}[lt]{7.5 cm}
	    \includegraphics[width=6cm]{bilder/stern-dreieck.png} 
	  \end{minipage}
	  \begin{minipage}[rt]{9.35 cm} %BASTEL!!
      \renewcommand{\arraystretch}{2}
	  \begin{tabular}{ll}
	Umwandlung $\triangle \rightarrow Y$: 
		&$Z_{c} = \dfrac{Z_{ac} Z_{bc}}{Z_{ab}+Z_{bc}+Z_{ac}}$\\
	Umwandlung $Y \rightarrow \triangle$: 
		&$Y_{ac}=\dfrac{Y_{a} Y_{c}}{Y_{a}+Y_{b}+Y_{c}}$\\
	Bei gleichen Widerst\"anden:
	&$R_Y = \frac{R_\triangle}{3}$ \\
	Bei gleichen Kapazit\"aten:
	&$C_Y = C_\triangle \cdot 3 $ \\
	Bei gleichen Induktivit\"aten:
	&$L_Y = \frac{L_\triangle}{3}$\\
	  \end{tabular}
      \renewcommand{\arraystretch}{1}
	  \end{minipage}
	%\end{figure}
	
	\subsection{Defektleistung von symmetrischen Drehstromverbrauchern}
		\begin{tabular}{| p{7.5cm} | l | l |}
			\hline
 				& Normalleistung		& Defektleistung \\
 				&&($P_{Def_{Str}}$ = Restleistung pro intaktem Strang) \\
 			\hline
	 			\textbf{Y-Schaltung bei Leiter- oder Strangausfall} ohne Neutralleiter &
	 				$P_{Norm} = 3 \cdot \frac{U^2_{Str}}{R} = \frac{U^2}{R}$ &
	 				$P_{Def} = \frac{1}{2} \frac{U^2}{R}$ \tiny{nur 50\% von $P_{Norm}$} \\
	 			&&\\
	 			\textbf{Y-Schaltung bei Leiter- oder Strangausfall} mit Neutralleiter &
	 				$P_{Norm} = 3 \cdot \frac{U^2_{Str}}{R} = \frac{U^2}{R}$ &
	 				$P_{Def_{Str}} = \frac{U^2_{Str}}{R} = \frac{1}{3} \frac{U^2}{R}$ \tiny{nur 33\% von
	 				$P_{Norm}$ pro Strang} \\ &&\\
	 			\textbf{$\Delta$-Schaltung bei Leiterausfall} &
	 				$P_{Norm} = 3 \cdot \frac{U^2}{R}$ &
	 				$P_{Def} = \frac{U^2}{R} + \frac{U^2}{2R} = \frac{3U^2}{2R}$ \tiny{nur 50\% von $P_{Norm}$} \\
	 			&&\\
	 			\textbf{$\Delta$-Schaltung bei Strangausfall} &
	 				$P_{Norm} = 3 \cdot \frac{U^2}{R}$ &
	 				$P_{Def_{Str}} = \frac{U^2}{R}$ \tiny{nur 33\% von $P_{Norm}$ pro Strang} \\
	 		\hline
		 \end{tabular} 
		 
	\subsection{Leistungsfaktorverbesserung}
		\subsubsection{... durch Zuschalten einer Wirkleistung P}
			\begin{tabular}{m{6cm} m{6cm}}
				\includegraphics[width=6cm]{bilder/Leistungsfaktorverbesserung_Zuschalten_Wirkleitsung.png}&
				$P_2 = \dfrac{Q_L}{\tan \varphi_2} - \dfrac{Q_L}{\tan \varphi_1}$
			\end{tabular}
		
		\subsubsection{... durch Zuschalten von Kondensatoren}
			\begin{tabular}{p{5cm} | p{5cm} p{6cm}}
				\textbf{Sternschaltung} &
				\textbf{Dreieckschaltung} \\
				\parbox[b][2.5cm][t]{5cm}{\includegraphics[width=4cm]{bilder/Leistungsfaktorverbesserung_Zuschalten_Kondensator_Stern.png}}
								&
				\parbox[b][2.5cm][t]{5cm}{\includegraphics[width=4cm]{bilder/Leistungsfaktorverbesserung_Zuschalten_Kondensator_Dreieck.png}}\\
				\multicolumn{2}{c}{
					$C = \dfrac{Q_c}{3 \cdot U_c^2 \cdot \omega}$
				}\\
				\multicolumn{2}{c}{
					$Q_c = P \cdot (\tan \varphi_1 - \tan \varphi_2) $ 
				} &
				\parbox{6cm}{
					$\varphi_1$ = Phasenwinkel unkompensiert\\
					$\varphi_2$ = Phasenwinkel kompensiert 
				} \\
				$U_c = \dfrac{U}{\sqrt{3}}$&
				$U_c = U$\\
				\multicolumn{2}{c}{
					$C_{\Delta} = \dfrac{C_Y}{3}$
				}			
				
			\end{tabular}
